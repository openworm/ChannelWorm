\documentclass[11pt]{article}
%%%%%%%%%%%%% General Package Imports %%%%%%%%%%%%%
\usepackage{amsmath,amsthm,amsfonts,amssymb,amscd,mathtools}
\usepackage[margin=3cm]{geometry}
\usepackage{float}
\usepackage{xcolor}
\usepackage[nottoc,notlot,notlof]{tocbibind}
\usepackage{titlesec}
\usepackage{hyperref}
\usepackage{algorithm}
\usepackage[noend]{algpseudocode}
%%%%%%%%%%%%%%%%%%%%%%%%%%%%%%%%%%%%%%%%%

%%%%%%%%%%%%% Macros and Custom Commands %%%%%%%%%%%
\newcommand{\comment}[1]{\textcolor{red}{\emph{{#1}}}}
\renewcommand{\vec}[1]{\mathbf{#1}}
\newcommand{\celegans}{\emph{C. elegans}}
\newcommand{\Celegans}{\emph{Caenorhabditis elegans}}
\newcommand{\bolditalic}[1]{\textbf{\emph{#1}}}
\newcommand{\channeltype}{\textbf{Channel Type:}}
\newcommand{\model}{\textbf{Model:}}
\newcommand{\parameters}{\textbf{Parameters to Fit:}}
%%%%%%%%%%%%%%%%%%%%%%%%%%%%%%%%%%%%%%%%%

%%%%%%%%%%%%% Document Parameters %%%%%%%%%%%%%%%%
% For package titlesec
\titlespacing\section{0pt}{12pt plus 0pt minus 2pt}{0pt plus 2pt minus 2pt}
\titlespacing\subsection{0pt}{12pt plus 0pt minus 2pt}{0pt plus 2pt minus 2pt}
\titlespacing\subsubsection{0pt}{12pt plus 0pt minus 2pt}{0pt plus 2pt minus 2pt}

% For package xcolor
\colorlet{shadecolor}{orange!15}

% For package geometry
\parindent 0in
\parskip 12pt
\geometry{margin=1in, headsep=0.25in}

% For package hyperref
\hypersetup{
    colorlinks,
    citecolor=black,
    filecolor=black,
    linkcolor=black,
    urlcolor=blue
}

% For package algspeudocode
\algdef{SE}[SUBALG]{Indent}{EndIndent}{}{\algorithmicend\ }%
\algtext*{Indent}
\algtext*{EndIndent}
\makeatletter
\def\BState{\State\hskip-\ALG@thistlm}
\makeatother

\setlength{\parskip}{.4cm}
%%%%%%%%%%%%%%%%%%%%%%%%%%%%%%%%%%%%%%%%%

\begin{document}

\thispagestyle{empty}

\begin{center}
{\LARGE \bf Genetic Algorithms for Fitting Hodgkin-Huxley Type Ion Channel Models}\\
\vspace{5pt}
\end{center}
\begin{center}
\vspace{-10pt}
\today
\end{center}

Here are some basic observations about the techniques in Gurkiewicz and Korngreen \cite{gurkiewicz2007numerical}.  The specifics are spelled out in the Methods section of their paper, but it took me some trial and error to fully understand the details, so I will try to make them explicit here.  The model optimized in the scripts in this repo is for the EGL-19
voltage gated calcium channel:

{\renewcommand{\arraystretch}{2}%
\begin{tabular}{| c | c |}
\hline
\textbf{Model Name} & EGL-19 \\ \hline
\channeltype & Voltage-gated calcium channel \\ \hline
\model & $I(V) = \underbrace{\big[\left.G_{max} \right/ 1 + e^{\frac{(V_{0.5} - V)}{k}}\big]}_{G_{0}} \cdot (V - V_{rev})$ \\
& $\frac{dG}{dt} = \frac{G_{0} - G}{\tau}$ \\ \hline
\parameters & $G_{max}$, $V_{0.5}$, $k$, $V_{rev}$ \\
\hline
\end{tabular}

Some clarifications that helped in implementing their techniques: 
\begin{itemize}
\item Each organism is characterized by a 4-tuple $(G_{max}, V_{0.5}, k, V_{rev})$. 
\item The initial population is created by randomly sampling from the parameters space.  Heuristically, 
a population size of 20 times greater than the number of parameters is sufficient.  I used a population size of 100
in the scripts here. 
\item The sample space does not need to be discretized.  
\item The tournament and subsequent generations are created as follows:
\begin{itemize}
\item Two sets of organisms are selected.  The best fit organism is selected from each pair.
\item A crossover operator is applied to the resulting pair where an index is randomly selected
and the corresponding values are swapped. 
\item \textbf{Adaptive sampling:} after a certain number of generations (1000 in this repo), future generations 
are selected in a more targeted manner.  After picking the highest scoring solution from \emph{all previous generations},
a new population is generated by randomly sampling from Gaussian distributions centered around the values of the
highest scoring solution with a variance of 5\%.  In other words, after the adaptive sampling procedure is started,
every successive generation is created from a \emph{single} descendent.  
\item There is no rigorous definition for the termination criterion.  I used that the variance of the past 1000 or so generations
was less than a fairly small threshold (for example $10^{-5}$.)
\end{itemize}
\item \textbf{Mutation operators:}
\begin{itemize}
\item Random Mutation: randomly sample from the entire parameter space. 
\item Random Gaussian drift: sample from a Gaussian distribution centered at the given parameter value with 5\% variance (the same function is used in the adaptive sampling method described above). 
\item Random Crossover: randomly select an index from two organisms and swap their entries.  
\end{itemize}

\end{itemize}








\bibliographystyle{ieeetr}
\bibliography{genetic_algorithm_notes}

\end{document}