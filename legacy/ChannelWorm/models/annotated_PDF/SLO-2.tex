%%%%%%%%%%%%%%%%%%%%%%%%%%%%%%%%%%%%%%%%%%%%%%%%%%%%%%%%%%%%%%%
%
% Welcome to Overleaf --- just edit your article on the left,
% and we'll compile it for you on the right. If you give 
% someone the link to this page, they can edit at the same
% time. See the help menu above for more info. Enjoy!
%
%%%%%%%%%%%%%%%%%%%%%%%%%%%%%%%%%%%%%%%%%%%%%%%%%%%%%%%%%%%%%%%
%
% For more detailed article preparation guidelines, please see:
% http://f1000research.com/author-guidelines

\documentclass[10pt,a4paper]{article}
\usepackage[procnames]{listings}
\usepackage{url}
\usepackage{color}
\usepackage[numbers]{natbib}

\definecolor{keywords}{RGB}{255,0,90}
\definecolor{comments}{RGB}{0,0,113}
\definecolor{red}{RGB}{160,0,0}
\definecolor{blue}{RGB}{0,0,160}
\definecolor{green}{RGB}{0,150,0}
\definecolor{verylightgray}{RGB}{240, 240, 240}

\usepackage[margin=3cm]{geometry}
\usepackage{float}
 
\lstset{language=Python, 
	backgroundcolor=\color{verylightgray},
        basicstyle=\ttfamily\footnotesize, 
        keywordstyle=\color{keywords},
        commentstyle=\color{comments},
        stringstyle=\color{blue},
        showstringspaces=false,
        identifierstyle=\color{green},
        procnamekeys={def,class},
        numbers=left,
		stepnumber=1,    
		firstnumber=1,
		numberfirstline=true,
        breaklines=true
}

%\renewcommand{\floatpagefraction}{.9}
%\renewcommand{\topfraction}{.9}
%\renewcommand{\dbltopfraction}{.9}
%\renewcommand{\bottomfraction}{.9}

\begin{document}

\begin{lstlisting}[
float=*,
floatplacement=!htbp,
language=XML,
caption={NeuroML File for SLO-2 Ion Channel}\label{lst:simple_test}]
<neuroml xmlns="http://www.neuroml.org/schema/neuroml2"  xmlns:xs="http://www.w3.org/2001/XMLSchema" xmlns:xsi="http://www.w3.org/2001/XMLSchema-instance" xsi:schemaLocation="http://www.neuroml.org/schema/neuroml2 https://raw.github.com/NeuroML/NeuroML2/development/Schemas/NeuroML2/NeuroML_v2beta4.xsd" id="ChannelWorm_SLO2_4_1">
    <ionChannelHH id="ChannelWorm_SLO2_4_1" conductance="10pS" species="K">
        
        <annotation>
            <rdf:RDF xmlns:rdf="http://www.w3.org/1999/02/22-rdf-syntax-ns#" xmlns:bqmodel="http://biomodels.net/model-qualifiers/" xmlns:bqbiol="http://biomodels.net/biology-qualifiers/">
                <!-- This is an ion channel model NeuroML2 file generated by ChannelWorm: https://github.com/openworm/ChannelWorm -->
                <rdf:Description rdf:about="ChannelWorm_SLO2_4_1">
                    <bqmodel:isDerivedFrom>
                        <rdf:Bag>
                            <rdf:li rdf:resource="ChannelWorm channel Name: SLO2 channel ID: 4, ModelID: 1"/>
                        </rdf:Bag>
                    </bqmodel:isDerivedFrom>
                    <bqmodel:isDescribedBy>
                        <rdf:Bag>
                            <!-- DOI: 10.1038/77670, PubMed ID: 10903569 
                                 SLO-2, a K+ channel with an unusual Cl- dependence. (Yuan A; Dourado M; Butler A; Walton N; Wei A; Salkoff L. Nat. Neurosci., 3(8):771-9) -->
                            <rdf:li rdf:resource="http://identifiers.org/pubmed/10903569"/>
                        </rdf:Bag>
                    </bqmodel:isDescribedBy>
                    <bqbiol:hasTaxon>
                        <rdf:Bag>
                            <!-- Known species: caenorhabditis elegans; taxonomy id: 6239 -->
                            <rdf:li rdf:resource="http://identifiers.org/taxonomy/6239"/>
                        </rdf:Bag>
                    </bqbiol:hasTaxon>
                </rdf:Description>
            </rdf:RDF>
        </annotation>

        <gateHHtauInf id="vda" instances="1">
            <timeCourse type="fixedTimeCourse" tau="0.000707626688985 s"/>
            <steadyState midpoint="-0.0774236324266 V" rate="1" scale="0.0307524845004 V" type="HHSigmoidVariable"/>
        </gateHHtauInf>
    </ionChannelHH>
</neuroml>


\end{lstlisting}

\end{document}