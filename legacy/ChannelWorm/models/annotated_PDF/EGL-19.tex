%%%%%%%%%%%%%%%%%%%%%%%%%%%%%%%%%%%%%%%%%%%%%%%%%%%%%%%%%%%%%%%
%
% Welcome to Overleaf --- just edit your article on the left,
% and we'll compile it for you on the right. If you give 
% someone the link to this page, they can edit at the same
% time. See the help menu above for more info. Enjoy!
%
%%%%%%%%%%%%%%%%%%%%%%%%%%%%%%%%%%%%%%%%%%%%%%%%%%%%%%%%%%%%%%%
%
% For more detailed article preparation guidelines, please see:
% http://f1000research.com/author-guidelines

\documentclass[10pt,a4paper]{article}
\usepackage[procnames]{listings}
\usepackage{url}
\usepackage{color}
\usepackage[numbers]{natbib}

\definecolor{keywords}{RGB}{255,0,90}
\definecolor{comments}{RGB}{0,0,113}
\definecolor{red}{RGB}{160,0,0}
\definecolor{blue}{RGB}{0,0,160}
\definecolor{green}{RGB}{0,150,0}
\definecolor{verylightgray}{RGB}{240, 240, 240}

\usepackage[margin=3cm]{geometry}
\usepackage{float}
 
\lstset{language=Python, 
	backgroundcolor=\color{verylightgray},
        basicstyle=\ttfamily\footnotesize, 
        keywordstyle=\color{keywords},
        commentstyle=\color{comments},
        stringstyle=\color{blue},
        showstringspaces=false,
        identifierstyle=\color{green},
        procnamekeys={def,class},
        numbers=left,
		stepnumber=1,    
		firstnumber=1,
		numberfirstline=true,
        breaklines=true
}

%\renewcommand{\floatpagefraction}{.9}
%\renewcommand{\topfraction}{.9}
%\renewcommand{\dbltopfraction}{.9}
%\renewcommand{\bottomfraction}{.9}

\begin{document}

\begin{lstlisting}[
float=*,
floatplacement=!htbp,
language=XML,
caption={NeuroML File for EGL-19 Voltage-Gated Calcium Channel.}\label{lst:simple_test}]
<?xml version="1.0" encoding="UTF-8"?>
<neuroml xmlns="http://www.neuroml.org/schema/neuroml2"
         xmlns:xsi="http://www.w3.org/2001/XMLSchema-instance"
         xsi:schemaLocation="http://www.neuroml.org/schema/neuroml2  https://raw.github.com/NeuroML/NeuroML2/master/Schemas/NeuroML2/NeuroML_v2beta.xsd"
         id="ca_boyle">

    <ionChannel id="ca_boyle" conductance="10pS" type="ionChannelHH" species="ca">

        <notes>Ca channel from Boyle and Cohen 2008</notes>

        <gateHHtauInf id="e" instances="2">

            <timeCourse type="fixedTimeCourse" tau="0.100027 ms"/>
            <steadyState type="HHSigmoidVariable" rate="1" scale="6.74821 mV" midpoint="-3.3568 mV"/>

        </gateHHtauInf>

        <gateHHtauInf id="f" instances="1">

            <timeCourse type="fixedTimeCourse" tau="150.88 ms"/>
            
            <!-- Note!!! 

                f gate is "inactivation" a/c B&C 2008 p172 
                but the scale value from Table A1 (kf) is positive (5mV), i.e. steady 
                state is zero for v << midpoint and 1 for v >> midpoint 
                Couple this with a very slow time course & the ion channel never 
                conducts as e^2 x f is always ~0 

                Using -5.03176mV NOT 5mV for k/scale to make this INACTIVATION 
                
                This is suggested by the value used here:
                https://github.com/openworm/muscle_model/blob/master/BoyleCohen2008/MatlabSupport/Main_Version/data/input.csv#L20
                
                see also
                https://github.com/openworm/muscle_model/blob/master/BoyleCohen2008/PythonSupport/Main_Version/compareToNeuroML2.py
            -->
            <steadyState type="HHSigmoidVariable" rate="1" scale="-5.03176 mV" midpoint="25.1815 mV"/>

        </gateHHtauInf>

        <customHGate id="h" instances="1" alpha="0.282473" ca_half="6.41889e-8 mM" k="-1.00056e-8 mM"/>

    </ionChannel>
\end{lstlisting}

\begin{lstlisting}[
float=*,
floatplacement=!htbp,
language=XML,
firstnumber=45],
    <ComponentType name="customHGate"
                   extends="gateHHtauInf"
                   description="Custom gate for h">

        <Parameter name="alpha" dimension="none"/>
        <Parameter name="k" dimension="concentration"/>
        <Parameter name="ca_half" dimension="concentration"/>


        <Constant name="SEC" dimension="time" value="1s"/>


        <Exposure name="tau" dimension="time"/>
        <Exposure name="inf" dimension="none"/>

        <Requirement name="caConc" dimension="concentration"/>

        <Dynamics>

            <DerivedVariable name="rateScale" exposure="rateScale" dimension="none" value="1"/>

            <DerivedVariable name="inf" dimension="none" exposure="inf" value="1 / (1 + (exp( (ca_half - caConc) / k)))"/>


            <DerivedVariable name="tau" dimension="time" exposure="tau" value="0 * SEC"/>
            <DerivedVariable name="q" exposure="q" dimension="none" value="inf"/>
            <DerivedVariable name="fcond" exposure="fcond" dimension="none" value="1 +((q-1) * alpha)"/>

        </Dynamics>
    </ComponentType>

</neuroml>
\end{lstlisting}


\end{document}